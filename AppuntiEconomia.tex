\documentclass{report}
\usepackage[utf8]{inputenc}
\usepackage[italian]{babel}
\usepackage[linktoc=all]{hyperref}
\usepackage{graphicx}
\usepackage{mathtools}
\usepackage{float}
\hypersetup{
	colorlinks,
	citecolor=black,
	filecolor=black,
	linkcolor=black,
	urlcolor=black
}
\renewcommand{\familydefault}{\sfdefault}
\usepackage[a4paper,includeheadfoot,margin=2.54cm]{geometry}
\usepackage{fancyhdr}
\newcommand\hr{\par\vspace{-.5\ht\strutbox}\noindent\hrulefill\par}
\usepackage{enumitem}


\begin{document}
	
	\author{Davide Parpinello}
	\title{%
			\begin{Large}
				Appunti di\\
			\end{Large}
		Economia ed Innovazione d'Impresa}
	\date{Luglio 2019}
	\maketitle
	
	\tableofcontents
	\renewcommand{\chaptermark}[1]{%
		\markboth{#1}{}}
	\addtocontents{toc}{\protect\hypertarget{toc}{}}
	\pagestyle{fancy}
	\fancyhf{}
	\rhead{\hyperlink{toc}{Economia ed Innovazione d'Impresa}}
	\lhead{\leftmark}
	\rfoot{\thepage}
	
	\chapter{Introduzione all'economia}
	\section{Il problema economico}
	Il problema economico fondamentale è la scarsità: le risorse a disposizione non sono mai sufficienti a soddisfare tutti i bisogni degli agenti economici.
	Questo implica l'esigenza di operare scelte. L'economia studia quindi le scelte degli agenti economici per gestire le risorse scarse e le regole e/o istituzioni per renderle migliori.
	\section{Input e output}
	Ogni società effettua scelte relative agli input: beni/servizi utilizzati e agli output: beni/servizi risultanti dai processi produttivi.
	\section{Gli economisti studiano..}
	\begin{itemize}
		\item Le scelte individuali: come gli agenti prendono decisioni mossi dal proprio self interest
		\item L'interazione tra gli agenti sul mercato
		\item Il sistema economico e il suo funzionamento nel complesso.
	\end {itemize}
	\section{Micro e macroeconomia}
	\begin{itemize}
		\item La microeconomia analizza il comportamento degli agenti e il funzionamento dei singoli mercati
		\item La macroeconomia considera l'economia come un sistema (es. ricchezza nazionale, inflazione..)
	\end {itemize}
	L'economia nasce nel '700 con un punto di vista macro per poi capire nel 1870 che il sistema dipende dal comportamento micro.
	\section{Scelte e agenti economici}
	\begin{itemize}
		\item Scelte degli individui o famiglie: cosa e quanto consumare, dove lavorare, quanto risparmiare. \textbf{Obiettivo:} massimizzare il benessere individuale.
		\item Scelte delle imprese: cosa, quanto e come produrre; a che prezzo vendere; come farsi pubblicità. \textbf{Obiettivo:} massimizzare il profitto.
		\item Scelte della collettività: come aggregare le preferenze individuali per soddisfare i bisogni collettivi. \textbf{Obiettivo:} massimizzare il benessere sociale.
	\end{itemize}
	In ogni caso i principi base del comportamento sono eguali per tutti gli agenti. Tutte le scelte dipendono dagli incentivi: motivazione misurabile.
	\section{Il mercato}
	Il mercato è l'istituzione principale dove ha luogo l'interazione economica degli agenti.\\
	Il mercato è un gioco a somma positiva: un meccanismo che favorisce tutti i partecipanti.\\
	Il funzionamento del mercato necessita di regole.
	\section{La concorrenza}
	La concorrenza può essere come processo naturale o come processo economico.
	\paragraph{Concorrenza come processo naturale} 
	Visione negativa: Darwinismo sociale, principio di sopravvivenza del più forte. Secondo tale visione il compito della società è quello di elevare l'uomo rispetto allo stato di natura quindi cercare di eliminare la concorrenza.
	\paragraph{Concorrenza come processo economico}
	Visione statica: processo che premia chi riesce a produrre al costo più basso.\\Visione dinamica: processo che premia chi riesce ad innovare e/o produrre il bene di qualità migliore.\\La concorrenza è quindi un processo positivo perché premia il merito di produttori e accresce il benessere dei consumatori. Occorrono comunque regole, per premiare chi merita davvero.
	\section{La scelta come \textit{trade-off}}
	\textit{Trade-off}: per ottenere una cosa si deve sempre rinunciare a qualcos'altro.\\Efficienza significa che la società ottiene il massimo possibile dalle proprie risorse scarse.\\Equità significa che i benefici che derivano dalle risorse di una società vengono distribuiti in modo giusto.\\Non è possibile ottenere entrambe le cose: esiste quindi un \textit{trade-off} rilevante nelle scelte del \textit{policy-maker}.
	\section{Efficienza}
	Il problema dell'efficienza è strettamente correlato con quello della scarsità e quindi riguarda tutti gli agenti economici.\\Efficienza infatti significa ottenere il massimo beneficio dalle risorse date oppure utilizzare il minimo ammontare di risorse per ottenere un dato livello di beneficio.\\Le regole del mercato servono a consentire il raggiungimento di queste condizioni.
	\section{Il costo opportunità}
	Ciò a cui si deve rinunciare ogni volta che si sceglie una determinata alternativa: ad esempio detenere il proprio potere d'acquisto in moneta invece che in titoli comporta la rinuncia ad un interesse.
	\section{La razionalità in economia}
	Essere razionali significa scegliere in base a un criterio e seguirlo coerentemente. Il criterio può essere uno qualsiasi, ma di solito in economia si adotta il criterio di massimizzazione della soddisfazione.
	\subsection{Razionalità come scelta al margine}
	Le variazioni marginali sono piccoli cambiamenti incrementali rispetto a una data quantità. Gli agenti razionali prendono le decisioni confrontando costi e benefici indotti da una variazione marginale.\\Per il criterio di scelta razionale, si compie una certa azione se e solo se il beneficio è maggiore del costo.
	\subsection{Gli individui rispondono agli incentivi}
	Un incentivo è un qualsiasi incremento del beneficio marginale o riduzione del costo marginale di una scelta. Al contrario, un disincentivo.\\Ogni variazione di costi/benefici induce una reazione razionale degli agenti, quindi il \textit{policy-maker} può indurre gli agenti a modificare certi comportamenti agendo sugli incentivi.
	\section{Perché scambiare?}
	Lo scambio consente di incrementare il benessere degli agenti attraverso la specializzazione. Lo scambio genera maggiore benessere per tutti e consente agli agenti di specializzarsi nelle attività che sanno svolgere meglio.
	\medskip \\
	Esistono due modi per soddisfare i bisogni di consumo:
	\begin{itemize}
		\item Scelta autarchica: si consuma solo ciò che si produce.
		\item Scelta della specializzazione e scambio: si consuma ciò che si ottiene in cambio di ciò che si è prodotto.
	\end{itemize}
	La più scelta è la seconda perché gli agenti specializzandosi in ciò che sanno fare e scambiando con altri agenti possono migliorare il proprio benessere. Quindi per incrementare il benessere occorre passare attraverso lo scambio.
	\subsection{Da cosa dipende la specializzazione}
	Intuitivamente dipende dalle differenze nei costi di produzione, misurabili come quantità di input necessaria per produrre un'unità di output (senso stretto) o come quantità di un bene a cui si deve rinunciare per produrre un'unità in più di un'altro bene (costo opportunità).
	\medskip \\ Abbiamo quindi due possibili criteri alla base dello scambio: vantaggio assoluto o vantaggio comparato.
	\section{Mercato ed efficienza}
	Un'economia di mercato è un sistema in cui gli agenti decidono liberamente cosa comprare, per chi lavorare, cosa produrre e chi assumere. L'interazione sul libero mercato degli agenti determina il massimo benessere possibile per l'intera collettività.
	\section{Modelli economici}
	\subsection{Modello del flusso circolare}
	Modo semplice di visualizzare le transazioni economiche che si realizzano tra gli agenti considerati, ovvero famiglie e imprese.\medskip \\Le transazioni avvengono attraverso il mercato dei beni e servizi e quello dei fattori produttivi.\medskip \\Evidenzia sia il flusso reale di beni, servizi e fattori che quello monetario di redditi e spese.
	\begin{figure}[h]
		\centering
		\includegraphics[width=0.7\linewidth]{images/flusso-circolare}
		\caption{flusso circolare}
		\label{fig:flusso-circolare}
	\end{figure}
	\subsection{Frontiera delle possibilità di produzione}
	Illustra i \textit{trade-off} che caratterizzano un sistema economico che produce solo due beni. Mostra la quantità massima di un bene che può essere prodotta, data la quantità dell'altro bene.
	\begin{itemize}
		\item \textbf{è decrescente}: per produrre una quantità maggiore di un bene è necessario sacrificare la produzione dell'altro bene;
		\item \textbf{è concava}: all'aumentare della produzione di un bene è necessario sacrificare quantità sempre crescenti dell'altro bene.
	\end{itemize}
	\chapter{Il mercato}
	\large Domanda, offerta ed equilibrio
	\section{Cos'è un mercato?}
	Tema centrale della microeconomia è lo studio del funzionamento dei mercati, ovvero insiemi regolati di compratori e venditori di beni, servizi o fattori produttivi.\medskip \\Una prima ipotesi istituzionale è che esista sempre un prezzo positivo per il bene, quindi venditori e compratori riusciranno sempre ad accordarsi per lo scambio. Ciò implica che il mercato sia l'istituzione dove gli agenti si scambiano beni.
	\medskip \\Sono presenti 4 forme di mercato:
	\begin{itemize}
		\item Concorrenza perfetta
		\item Concorrenza monopolistica
		\item Monopolio
		\item Oligopolio
	\end{itemize}
	\section{Mercato di concorrenza perfetta}
	Vengono soddisfatte 4 ipotesi:
	\begin{enumerate}
		\item \textbf{molteplicità e free entry:} presenti molti compratori e venditori e nessun vincolo all'ingresso nel mercato
		\item \textbf{assenza di potere di mercato:} nessun partecipante riesce a controllare prezzi o quantità
		\item \textbf{uniformità del prodotto:} il prodotto è omogeneo
		\item \textbf{informazione perfetta:} tutti i partecipanti conoscono tutte le informazioni relative al mercato. Le informazioni devono essere simmetriche e complete.
	\end{enumerate}
	Da queste ipotesi discende:
	\begin{itemize}
		\item \textbf{Legge del prezzo unico:} nel mercato vige un unico prezzo
		\item \textbf{Comportamento price taking:} compratori e venditori subiscono il prezzo di mercato
	\end{itemize}
	\section{La domanda di mercato}
	\begin{itemize}
		\item La \textit{quantità domandata} è l'ammontare di beni che i compratori vogliono e possono acquistare
		\item La \textit{scheda di domanda} è una tabella che mostra la relazione tra il prezzo del bene e la quantità domandata
		\item La \textit{curva di domanda} è la linea discendente che mette in relazione prezzo e quantità in un sistema di assi cartesiani
	\end{itemize}
	\subsection{Le determinanti della domanda}
	La domanda di mercato dipende da diversi fattori, tra cui prezzo di mercato, reddito dei consumatori, prezzo dei beni collegati, gusti e aspettative dei consumatori.
	\medskip \\Un bene normale è un bene la cui domanda aumenta al crescere del reddito, mentre un bene inferiore è un bene la cui domanda, al contrario, diminuisce al crescere del reddito.
	\medskip \\Quando una diminuzione del prezzo di un bene riduce la domanda di un'altro essi si dicono sostituti, ad esempio treno e aereo; quando invece la diminuzione del prezzo di uno aumenta la domanda di un'altro vengono detti complementari.
	\section{La curva di offerta}
	\begin{itemize}
		\item La \textit{quantità offerta} è l'ammontare di bene che i venditori possono e vogliono vendere
		\item La \textit{scheda di offerta} è una tabella che mostra la relazione tra il prezzo del bene e la quantità offerta
		\item La \textit{curva di offerta} è la linea ascendente che mette in relazione il prezzo e la quantità offerta
	\end{itemize}
	\subsection{Le determinanti dell'offerta}
	L'offerta di mercato di un bene dipende da numerosi fattori, tra cui il prezzo di mercato, la tecnologia di produzione, il prezzo e la dotazione dei fattori di produzione, il numero di produttori/venditori, \underline{le aspettative degli imprenditori}.
	\section{Equilibrio di mercato}
	In un mercato l'equilibrio è determinato dall'intersezione tra curva di domanda e curva di offerta (\textit{forbice marshalliana}). Il prezzo di equilibrio è il prezzo che uguaglia domanda e offerta, mentre la quantità di equilibrio è la quantità che eguaglia domanda e offerta.
	\begin{figure}[h]
		\centering
		\includegraphics[width=0.7\linewidth]{images/forbice-marshalliana}
		\caption[Forbice marshalliana]{Forbice marshalliana}
		\label{fig:forbice-marshalliana}
	\end{figure}
	\section{Il disequilibrio}
	Si ha un eccesso di offerta quando la quantità offerta è maggiore di quella domandata, ovvero quando il prezzo di mercato è più alto del prezzo di equilibrio e quindi i produttori non riescono a vendere tutto a quel prezzo.
	\medskip \\
	SI ha invece un eccesso di domanda quando la quantità domandata è maggiore di quella offerta, ovvero quando il prezzo di mercato è più basso del prezzo di equilibrio e quindi i consumatori non riescono ad acquistare tutto a quel prezzo.
	\subsection{Due tipi di aggiustamento}
	\subsubsection{Approccio walrasiano}
	Il disequilibrio è differenza nelle quantità e segnala la presenza di un prezzo troppo alto/basso. Si raggiunge l'equilibrio variando il prezzo.
	\subsubsection{Approccio marshalliano}
	Il disequilibrio è differenza tra prezzo d'acquisto e prezzo di vendita e segnala la presenza di una quantità offerta troppo alta/bassa. Si arriva all'equilibrio variando la quantità.
	\medskip \\L'approccio walrasiano è più semplice ma ha il problema istituzionale di chi fissa il prezzo di partenza; l'approccio marshalliano è invece più realistico.
	\section{La clausola \textit{ceteris paribus}}
	Espressione impiegata dagli economisti per indicare che in una certa analisi tutte le variabili diverse da quelle in oggetto sono ipotizzate costanti. Clausola alla base del metodo di equilibrio parziale, dove si studia un singolo mercato in isolamento
	
	\chapter{Il monopolio}
	\section{Introduzione al monopolio}
	\subsection{Caratteristiche del monopolio}
	Il monopolio è la forma di mercato agli antipodi della PC (perfettamente concorrenziale). Un'impresa è monopolista se è l'unica che vende un certo prodotto, se il prodotto non ha dei buoni sostituti e se non esiste possibilità di entrata nel mercato.
	\medskip \\Il monopolista è quindi price-maker, ha potere di mercato sul prezzo. I veri monopoli sono rari perché è raro che vi siano prodotti davvero unici, quindi il monopolio puro è considerato un caso ideale.
	\subsection{Perché esiste il monopolio?}
	La causa fondamentale è la presenza di tre barriere all'entrata sul mercato:
	\begin{enumerate}
		\item \underline{barriere di tipo oggettivo:} solo chi possiede un determinato input può produrre un certo bene
		\item \underline{barriere di tipo legale:} brevetti, marchi, copyright.. indispensabili per incentivare le imprese a innovare
		\item \underline{barriere di tipo economico:} presenza di forti economie di scala.
	\end{enumerate}
	\subsection{Il monopolio naturale}
	Si ha un monopolio naturale quando una singola impresa può fornire un certo bene all'intero mercato ad un costo inferiore di altre imprese. Succede nel caso in cui la dimensione efficiente di un'impresa è così grande che in quel settore solo un'impresa può fornire il prodotto al mercato al minimo costo medio.
	\medskip \\Solo l'aumento della domanda può eliminare il monopolio naturale.
	\subsection{Monopolio \textit{versus} concorrenza perfetta}
	Nel monopolio esiste un unico produttore la cui domanda coincide con la domanda di mercato, che agisce da price-maker ottenendo extra profitto e il cui comportamento è vincolato solo dalla domanda.
	\medskip \\In un mercato PC esistono molte imprese la cui curva di domanda è orizzontale, che agiscono da price-takers e che al prezzo dato possono vendere qualsiasi quantità.
	\subsection{La massimizzazione del profitto del monopolista}
	Il monopolista massimizza il profitto seguendo la regola che il ricavo medio sia pari al costo medio, per determinare la quantità ottimale. Il prezzo a cui vende sarà sempre maggiore del CM, ottenendo extra profitto che permane anche nel lungo periodo.
	\subsection{Politiche pubbliche anti-monopolio}
	Il policy-maker può evitare il monopolio mediante leggi e autorità antitrust, ad esempio:
	\begin{itemize}
		\item impedendo che la fusione tra più imprese crei un nuovo monopolio; \item imporre il comportamento ai monopolisti ad esempio riguardo al prezzo; \item nazionalizzando i monopoli privati; \item non facendo nulla: il mercato elimina da solo i monopoli.
	\end{itemize}
	\subsection{La discriminazione di prezzo}
	Si intende la possibilità per il monopolista di violare la legge di prezzo unico, vendendo lo stesso bene a prezzo diverso a clienti diversi; allo stesso cliente a prezzi diversi in base alla quantità; un mix tra le due: in base a cliente e quantità acquistata.
	\medskip \\
	La discriminazione di prezzo è impossibile in un mercato PC: è necessario avere potere di mercato.
	\medskip \\
	Per esercitare ciò devono valere due condizioni:
	\begin{itemize}
		\item il monopolista deve avere informazioni per poter suddividere i clienti in base alla loro disponibilità a pagare;
		\item non devono esistere possibilità di arbitraggio: rivendere con profitto il bene a chi, comprando dal monopolista, dovrebbe pagare di più.
	\end{itemize}
	Si parla invece di discriminazione perfetta quando il monopolista si appropria dell'intero surplus del consumatore applicando ad ogni cliente un prezzo pari alla sua disponibilità a pagare.
	\section{Concorrenza monopolistica}
	La concorrenza monopolistica (MC) è una forma di mercato intermedia tra PC e monopolio.
	\medskip \\Le caratteristiche principali sono:
	\begin{itemize}
		\item la presenza di molti venditori, che competono per accaparrarsi gli stessi clienti;
		\item la differenziazione del prodotto: ogni impresa produce un prodotto che differisce almeno in parte da quello delle altre imprese, quindi ogni impresa fronteggia una curva di domanda specifica;
		\item libertà di entrata e uscita: non ci sono restrizioni.
	\end{itemize}
	\subsection{L'impresa MC nel breve periodo}
	L'impresa segue la stessa regola di massimizzazione del profitto del monopolista, poiché nel breve periodo non esiste concorrenza per quella particolare varietà del prodotto.
	\medskip \\Quindi l'impresa ottiene extra-profitti positivi e il benessere sociale non è massimizzato.
	\subsection{L'ingresso di nuove imprese}
	L'ottenimento di extra-profitti positivi incoraggia l'ingresso di nuove imprese, che producono diverse varietà del prodotto, aumentando quindi il numero di prodotti offerti e riducendo la domanda per le altre imprese.\medskip \\In caso di perdite si avrà l'uscita di alcune imprese e quindi l'aumento della domanda; al ridursi della domanda l'extra profitto si riduce a zero.
	\subsection{L'equilibrio di lungo periodo}
	Le imprese entrano ed escono dal mercato MC finché gli extra profitti non divengono zero.\medskip \\
	Nel lungo periodo, come nel monopolio, il prezzo di equilibrio eccede il costo medio, perché bisogna che RM = CM ma la pendenza negativa della curva di domanda implica che RM sia comunque inferiore al prezzo.\medskip \\Inoltre come nel mercato PC, il prezzo uguaglia il Costo Medio Totale: la libertà di entrata e uscita fa si che l'equilibrio di lungo periodo possa aversi solo in assenza di extra profitti.
	\subsubsection{Capacità in eccesso}
	Due differenze notevoli negli equilibri di lungo periodo tra MC e PC sono la capacità in eccesso e il \textit{mark-up}.
	\medskip \\Nella PC non c'è alcuna capacità produttiva in eccesso: ciascuna impresa PC produce la quantità efficiente (CMeT minimo).
	\medskip \\In MC nel lungo periodo si ha un eccesso di capacità produttiva: l'output è minore della quantità efficiente.
	\subsection{Pubblicità e marchi}
	La possibilità di ottenere extra-profitti è ciò che spinge le imprese a fare pubblicità e utilizzare marchi: sono infatti strumenti con cui possono differenziare il prodotto.
	\medskip \\Sono strumenti per risolvere il problema dell'asimmetria informativa tra le imprese, che offrono prodotti differenziati.
	\section{Oligopolio}
	\subsection{Un nuovo tipo di razionalità}
	Fino ad ora le scelte degli altri non erano rilevanti: gli altri sono singolarmente irrilevanti perché troppo piccoli rispetto al mercato, in PC e MC, mentre gli altri semplicemente non esistono, nel monopolio. Le scelte erano quindi in un ambiente parametrico.
	\medskip \\
	Quando invece esistono altri, le cui scelte possono quindi influenzare le nostre decisioni, si passa ad una razionalità non parametrica o strategica, secondo il concetto di interdipendenza.
	\subsection{Caratteristica dell'oligopolio}
	Un oligopolio è un mercato in cui esistono solo poche imprese, che offrono prodotti identici o simili.\medskip \\La caratteristica fondamentale del mercato è l'interdipendenza: data l'esistenza di poche imprese, le azioni di ciascuna hanno un effetto rilevante sull'esito del mercato per tutte le altre.\medskip \\La concorrenza quindi è tale ovvero cercare di battere le imprese rivali, solitamente è un problema di strategia: si utilizza la teoria dei giochi al posto delle curve.
	\subsubsection{La teoria dei giochi}
	è la teoria matematica che studia il comportamento razionale in condizioni di interdipendenza strategica, cioè quando la scelta di quale azione intraprendere deve tenere conto delle scelte e delle reazioni degli altri agenti.
	\subsection{Il caso più semplice: il duopolio}
	Il duopolio è un oligopolio con solo due imprese.
	\medskip \\
	L'oligopolio determina una situazione strategica, in cui le decisioni delle imprese devono tener conto dell'interdipendenza con le scelte delle imprese rivali.
	\subsection{Collusione}
	Una delle possibilità è cooperare con le rivali e agire tutte insieme come fossero un unico monopolista, formando un \textbf{monopolio congiunto}. Questo comportamento di cooperazione conviene, il problema è che, una volta concluso l'accordo, ciascuna impresa ha un incentivo a deviare unilateralmente l'accordo, causando la rottura del cartello.
	\medskip \\Il processo di deviazione si arresta quando entrambe producono una quantità tale che nessuna ha l'incentivo a deviare, raggiungendo l'equilibrio.
	\subsubsection{Equilibrio di Nash}
	Una situazione in cui nessun agente ha un incentivo a deviare, è un concetto di razionalità individuale molto generale. Per un agente la scelta razionale è quella da cui non si ha motivo di deviare unilateralmente.
	\medskip \\Si applica a situazioni di interazione strategica, cioè quando un individuo, per prendere la migliore decisione, deve considerare i possibili comportamenti degli altri.
	\subsection{L'esito di un mercato oligopolistico}
	Indipendentemente dalle norme antitrust e senza un efficace meccanismo vincolante, gli accordi collusivi non reggono: ogni impresa ha un incentivo a deviare. Pertanto, l'esito sarà un profitto inferiore al monopolio ma maggiore di PC.
	\medskip \\Se esiste invece un meccanismo vincolante, che obbliga a rispettare l'accordo, l'esito coincide con quello di monopolio. Però è evidente che più sono le imprese più difficile è rispettare l'accordo.
	\subsection{La politica economica e l'oligopolio}
	La collusione è socialmente desiderabile per gli oligopolisti, ma non per la società dato che ha esito identico al monopolio; per il benessere sociale è meglio che gli oligopolisti competano e pervengano all'equilibrio di Nash.
	\medskip \\Le norme antitrust vietano accordi tra imprese volti a spartirsi il mercato o raggiungere monopolio. Può succedere però che non esista un vero accordo dietro un comportamento collusivo ma ci sia solamente l'applicazione della razionalità economica da parte delle imprese, che collaborano senza un esplicito accordo.
	\subsubsection{Due problemi per l'antitrust}
	Se la collusione non è un equilibrio neppure nel semplice caso di duopolio, a cosa servono i divieti antitrust in merito?
	\medskip \\Se un comportamento collusivo scaturisce dal ragionamento delle singole imprese, senza accordi tra esse, il diritto antitrust interviene o no?
	
	\chapter[Scelte dell'imprenditore]{Scelte dell'imprenditore: produzione e costi}
	\section{Le scelte dell'imprenditore}
	Gli imprenditori devono prendere decisioni ed effettuare scelte razionali (minimizzare i costi e massimizzare i ricavi). Queste scelte si basano su previsioni, pianificando il livello di produzione e scegliendo la combinazione dei fattori produttivi rispettando vincoli tecnici (tecnologia) e di mercato (domanda e prezzo).
	\medskip \\
	Queste scelte possono riguardare:
	\begin{enumerate}
		\item La grandezza: dipende dalla possibilità di collocare i prodotti sul mercato, viene preceduta dalle previsioni della futura domanda;
		\item La dislocazione dell'impresa: riguarda la scelta del luogo dove sorge, segue il criterio della convenienza economica e efficienza produttiva;
		\item La tecnica produttiva: definizione delle modalità del processo produttivo, da questa dipende il rapporto tra capitale fisso e variabile.
	\end{enumerate}
	La capacità produttiva può variare solo mediante investimenti, ovvero aumentando il capitale variabile (es. beni intermedi). Bisogna scegliere la tecnica produttiva che consente di produrre a costi più bassi.
	\section{La funzione di produzione}
	Relazione che intercorre tra input utilizzati nel processo produttivo e la quantità di prodotto finale.
	\[ fdp: Q = F(input1, input2, input3 ...) \]
	La sua forma dipende dalla tecnologia. All'imprenditore non interessa cosa avviene davvero dentro, per lui conta solo che Q sia ottenuto al minimo costo.
	\section{Prodotto marginale decrescente}
	\begin{itemize}
		\item Prodotto medio PMe: rapporto tra prodotto totale e quantità utilizzata di un certo fattore di produzione.
		\item Prodotto marginale PM: incremento di prodotto ottenuto incrementando di un'unità un solo altro fattore, a parità degli altri\[ PM_{i} = \Delta Q / \Delta input_{i} \]
		\item Principio del prodotto marginale decrescente: al crescere della quantità utilizzata di un certo fattore il suo prodotto marginale diminuisce.
	\end{itemize}
	\section{L'impresa multi prodotto e la Frontiera delle possibilità di produzione}
	Quando un'impresa produce più di un prodotto l'imprenditore deve risolvere un terzo problema: data tecnologia e fattori di produzione, come distribuirli tra i diversi processi produttivi in modo efficiente?
	\medskip \\
	Una risposta è data dalla Frontiera delle possibilità di produzione (FPP), una funzione che racchiude le diverse combinazioni efficienti di prodotti che un'impresa può produrre, dati fattori e tecnologia.
	\subsection{La massimizzazione del profitto}
	L'ipotesi fondamentale è che l'impresa decida quanto produrre avendo come \underline{obiettivo} la massimizzazione del profitto.
	\section{I costi di produzione e il profitto}
	I costi di produzione si dividono in costi espliciti, che richiedono un esborso monetario, ed impliciti, che non richiedono esborso (costi opportunità).
	\medskip \\
	Quando i ricavi superano la somma dei costi si parla di profitto puro, mentre la differenza tra ricavi e costi espliciti è detta profitto contabile (non interessa agli economisti).
	\subsection{Costi fissi e variabili}
	I costi di produzione si dividono in:
	\begin{itemize}
		\item \textbf{Costi fissi totali CF:} non variano con l'ammontare di output prodotto (es. capannone);
		\item \textbf{Costi variabili totali CV:} variano con l'output (es. materie prime).
	\end{itemize}
	L'essere fissi o variabili dipende dal periodo di tempo considerato. La durata dei periodi è economica, un periodo è lungo se tutti i costi sono variabili.
	\subsection{Costo marginale}
	Incremento del costo totale necessario per produrre un'unità addizionale di output. Le scelte dell'impresa dipendono dal confronto tra questo e il ricavo marginale.
	\subsection{Relazione tra costi medi e marginali}
	Quando il costo marginale è minore del CMeT, quest'ultimo diminuisce.
	\medskip \\
	Quando il costo marginale è maggiore del CMeT, questo aumenta.
	\medskip \\La dimensione efficiente di un'impresa è la quantità di output per cui il CMeT è minimo.
	\section{Economie di scala}
	Quando i costi medi di lungo periodo diminuiscono all'aumentare della produzione si verificano economie di scala: reali (o tecniche) se sono associate a variazioni delle quantità di input, pecuniarie se associate a variazioni dei prezzi pagati dall'impresa per i fattori di produzione.
	\medskip \\
	Possono essere realizzate a livello di stabilimento o di impresa nel complesso. Quando si aumenta la scala di produzione si può beneficiare della specializzazione dividendo il lavoro o specializzando il management.
	\subsection{Economie di scala reali (tecniche)}
	\begin{itemize}
		\item \textbf{Produzione su larga scala:} producendo grandi volumi si possono utilizzare impianti non possibili a piccoli imprenditori
		\item \textbf{Indivisibilità degli input di capitale e lavoro:} non si possono comprare macchinari in parte. Alcuni input non aumentano se aumenta la scala di produzione.
		\item \textbf{Economie di apprendimento:} lavoratori e manager diventano esperti
		\item \textbf{Relazioni geometriche} tra input e output abbassano i costi quando aumenta la scala di produzione
	\end{itemize}
	\subsection{Economie di scala pecuniarie}
	\begin{itemize}
		\item \textbf{Raccolta di capitali:} una grande impresa può offrire ai prestatori maggiori garanzie di una piccola. Può avere accesso a fonti di finanziamento maggiori.
		\item \textbf{Economie di acquisto e commercializzazione:} i fornitori offrono sconti per ordini su scala vasta. Grandi imprese beneficiano di forme di pubblicità su larga scala.
		\item \textbf{Economie di trasporto:} grandi imprese possono mettere in esercizio impianti di produzione e vendita in regioni differenti.
	\end{itemize}
	\subsection{Economie di scopo (o di produzione congiunta)}
	Sono i minori costi ottenuti da un'impresa quando produce due o più prodotti usando le stesse risorse, ad esempio acquistando grandi quantità di input per prodotti diversi, ripartendo costi come finanza o marketing, riutilizzando i dati sui clienti per strategie di marketing.
	\section{Diseconomie di scala}
	Quando i costi medi di lungo periodo crescono all'aumentare della produzione si verificano diseconomie di scala: sono causate da sbagli nella funzione manageriale (distorsioni comunicative o comportamenti opportunistici), lunghe catene di comando, mancanza di interesse della forza lavoro o relazioni industriali problematiche.
	\medskip \\
	Causa di diseconomie di scala possono essere anche i costi di trasporto che aumentano, o diseconomie esterne di scala come l'espansione di un impianto produttivo.
	\subsection{Valutazione della tecnica produttiva più conveniente}
	L'imprenditore deve considerare anche i costi monetari, conoscere i prezzi dei fattori produttivi e individuare la tecnica che minimizza i costi di produzione.
	\medskip \\Esistono due tecniche: a \underline{più alta intensità di lavoro}, o a \underline{più alta intensità di capitale}.
	\subsection{Isoquanti e isocosti}
	L'\textbf{isoquanto} è la relazione tra tutte le combinazioni di lavoro e capitale che possono essere realizzate per produrre l'output.
	\medskip \\Le funzioni di \textbf{isocosto} mostrano le combinazioni di K e L impiegabili da parte dell'impresa aventi lo stesso costo totale.
	
	\chapter{Le barriere all'entrata}
	\section{Introduzione}
	Le barriere all'entrata sono le condizioni che permettono alle imprese già presenti nel mercato di ottenere extraprofitti, oppure sono considerate come un costo di produzione che un'impresa deve sostenere per entrare nel mercato non gravando sulle altre.
	\medskip \\Un fattore che determina l'entrata sono le barriere all'uscita, ad esempio investimenti non recuperabili.
	\medskip \\Tipi di barriere all'entrata sono
	\begin{itemize}
		\item Economie di scala
		\item Vantaggio assoluto di costo
		\item Differenziazione di prodotto
		\item Costi di cambiamento
		\item Esternalità di rete
		\item Barriere legali
		\item Barriere geografiche
	\end{itemize}
	\section{Differenziazione di prodotto}
	All'entrata, preferenza dei consumatori di prodotti già esistenti sul mercato rispetto a quelli delle nuove imprese. Una barriera è data dai clienti fedeli ai marchi e alla reputazione di imprese esistenti.
	\medskip \\
	Le barriere possono presentarsi come spesa maggiore in pubblicità per ogni cliente, investimenti richiesti per la campagna elevati e rischiosi.
	\medskip \\
	Le imprese già esistenti possono aumentare il grado di differenziazione del prodotto attraverso la proliferazione delle marche.
	\section{Costi di cambiamento o di riconversione}
	I clienti incorrono in costi di cambiamento se cambiando il fornitore devono sostenere costi addizionali come costi di ricerca di informazioni, costi sull'impiego del nuovo prodotto, costi di rinuncia a un'assistenza post-vendita per riparazioni, costi psicologici derivanti dalla rescissione di un rapporto consolidato.
	\medskip \\
	Questi costi innalzano barriere all'entrata per prodotti come carte di credito o cellulari. La strategia che prevede inizialmente offerte a prezzi bassi per catturare i clienti per poi rialzarli funziona meglio quando i consumatori già catturati possono essere separati dai nuovi, ad esempio quando sono costretti ad acquistare prodotti complementari.
	\medskip \\
	Se i costi di riconversione sono alti, sono necessari notevoli vantaggi economici per i nuovi entranti.
	\section{Esternalità di rete}
	Sorgono quando il valore di un prodotto per un consumatore dipende dal numero degli altri consumatori. Possono essere dirette, quando all'aumentare dei clienti il prodotto diventa attraente per gli altri, o indirette, quando l'adozione più ampia influenza un mercato collegato (ad esempio sistema operativo e software compatibili).
	\medskip \\
	Nel caso di standard tecnologici compatibili, i consumatori possono godere di tutti i vantaggi della rete senza rivolgersi ad un unico produttore. Tuttavia un'impresa dominante potrebbe ancora preferire l'incompatibilità perché genera barriere all'entrata.
	\section{Barriere legali}
	Erette dagli stati e imposte per legge, ad esempio autorizzazione per le attività, diritti di monopolio, brevetti o politiche pubbliche.
	\section{Barriere geografiche}
	Restrizioni imposte alle imprese estere che tentano di operare nel mercato interno di una nazione. Possono essere fisiche (frontiera), tecniche (conformità agli standard), fiscali (dazi o quote), politiche preferenziali o barriere linguistiche e culturali.
	\medskip \\Per la scuola di Chicago, è rilevante non l'esistenza di barriere ma la velocità con la quale possono essere superate.
	\section{Barriere di tipo finanziario}
	Per l'ingresso potrebbero servire ingenti risorse finanziarie, ad esempio capitale anche per sostenere il credito alla clientela, le scorte e per assorbire le perdite di avviamento. \medskip \\L'entità dell'investimento richiesto e il rischio collegato dovrebbero dissuadere il potenziale entrante.
	\section{Difficoltà di accesso ai canali distributivi}
	La rete distributiva tipica potrebbe essere presidiata dalle imprese già esistenti costringendo i nuovi entranti a diminuire il prezzo o investire in pubblicità.\medskip \\Le imprese potrebbero avere rapporti esclusivi con la rete distributiva, tali da creare una barriera così alta da esserci bisogno di una nuova rete.
	\section{Strategie di diffusione all'entrata}
	Le barriere all'entrata strategiche sono quei comportamenti attuati dalle imprese già presenti al fine di escludere o mettere in difficoltà nuovi rivali, ad esempio usare prezzi limite o predatori, o differenziare il prodotto o le marche.
	
	\chapter{Impresa e società}
	\section{La scelta della forma giuridica}
	La scelta di una tra le forme giuridiche deve tenere conto di:
	\begin{itemize}
		\item Numero dei promotori;
		\item Natura dell'attività svolta;
		\item Dimensione dell'impresa;
		\item Disponibilità di capitali;
		\item Grado di responsabilità che i soci intendono assumere.
	\end{itemize}
	\section{Il soggetto giuridico in diverse aziende}
	\paragraph{Nelle famiglie}
	Il s.g. di responsabilità è il capofamiglia, mentre quello di rappresentanza può comprendere anche altri soggetti.
	\paragraph{Nelle imprese individuali}
	Il s.g. di responsabilità è l'imprenditore individuale il cui nome è registrato nell'Ufficio del registro delle imprese, mentre il s.g. di rappresentanza sono anche altri soggetti che possono assumere rapporti giuridici (es. il direttore generale).
	\paragraph{Nelle imprese in forma di società}
	Il s.g. di responsabilità corrisponde alla società stessa o ai soci amministratori, mentre il s.g. di rappresentanza comprende gli amministratori e coloro che hanno potere di gestione.
	\paragraph{Nelle aziende composte pubbliche}
	Il s.g. di responsabilità è l'Ente Pubblico Territoriale mentre il s.g. di rappresentanza sono gli organi che hanno potere di impegnata in rapporti giuridici (es. Sindaco, segretario comunale..)
	\begin{figure}[h]
		\centering
		\includegraphics[width=0.7\linewidth]{images/imprenditore-impresa-societa}
		\caption{Imprenditore, impresa e società: definizioni}
		\label{fig:imprenditore-impresa-societa}
	\end{figure}
	\section{Tipi di società}
	\paragraph{Società di persone}Le S.S., le S.N.C. e le S.A.S. dove i soci assumono rilevanza come persone fisiche.
	\paragraph{Società di capitali}Le S.P.A., le S.A.A. e le S.R.L. dove la persona del socio non assume rilevanza diretta se non come membro degli organi sociali.
	\subsection{Società di fatto}
	Società per le quali non è stato formalmente stipulato l'atto costitutivo.
	\subsection{Società irregolari}
	Società nelle quali non si è perfezionata la formale costituzione.
	\subsection{Società semplice (S.S)}
	L'atto costitutivo non è soggetto a forme speciali e viene registrato nella Sezione Speciale del Reg. Imprese.\medskip \\La partecipazione agli utili e alle perdite è proporzionale ai conferimenti.
	\subsection{Società in nome collettivo (S.N.C)}
	Tutti i soci rispondono solidalmente e illimitatamente per le obbligazioni sociali. Atto costitutivo da depositare entro 30gg al Reg.Imp. Salvo diversi accordi l'amministrazione spetta disgiuntamente a ciascun socio.
	\subsection{Società in accomandita semplice (S.A.S.)}
	Le categorie di soci sono accomandatari e accomandanti. Le quote di partecipazione non sono rappresentate da azioni.\medskip \\Gli accomandatari rispondono solidalmente e illimitatamente per le obbligazioni sociali; gli accomandanti limitatamente alla quota conferita.\medskip \\
	L'amministrazione spetta soltanto agli accomandatari.
	\subsection{Società a responsabilità limitata (S.R.L)}
	La responsabilità dei soci è limitata ai conferimenti attuati. Per le obbligazioni sociali risponde solo la società con il patrimonio. Le quote di partecipazione non sono rappresentate da azioni.\medskip \\Si costituisce per atto pubblico e acquista personalità giuridica con l'iscrizione al Reg. Imp.
	\subsection{Società per azioni (S.P.A.)}
	La responsabilità dei soci è limitata ai conferimenti attuati. Per le obbligazioni sociali risponde solo la società con il patrimonio. Le quote di partecipazione sono rappresentate da azioni.
	\medskip \\Si costituisce per atto pubblico e acquista personalità giuridica con l'iscrizione al Reg. Imp.\medskip \\Organi della società sono l'assemblea dei soci, gli amministratori e il collegio sindacale.
	\subsection{Società in accomandita per azioni (S.A.A.)}
	Le categorie di soci sono accomandatari e accomandanti. Le quote di partecipazione sono rappresentate da azioni.\medskip \\Gli accomandatari rispondono solidalmente e illimitatamente per le obbligazioni sociali; gli accomandanti limitatamente alla quota conferita.
	\section{Il soggetto economico}
	\paragraph{Nell'azienda domestico-patrimoniale} Il s.e. è il capofamiglia e gli altri membri capaci di influenzarlo.
	\paragraph{Nell'impresa individuale}Corrisponde all'imprenditore che assume anche la figura di soggetto giuridico.
	\paragraph{Nella grande impresa}
	Nelle S.P.A. sono i titolari del capitale di comando, coloro che hanno conferito una parte del capitale sociale tale da avere la maggioranza nelle assemblee. Ne possono far parte a volte anche i lavoratori, come gruppo organizzato sindacalmente.
	\medskip \\Rientra in esso anche il soggetto operativo, e in condizioni patologiche (impresa insolvente) anche i creditori.
	\paragraph{Nelle aziende composte pubbliche}Costituito dal soggetto operativo e dalla comunità sociale che ne usufruisce i servizi.
	\subsection{Gli stakeholder}
	Sono soggetti esterni all'impresa portatori di interessi nei suoi confronti. Possono essere sogg. economici e non.\medskip \\Alcuni esempi sono i dipendenti, i fornitori, i concorrenti, i clienti, organizzazioni, enti o governi.
	
	\chapter{Introduzione alla macroeconomia}
	\paragraph{I dati} I dati sull'economia italiana sono raccolti da enti come ISTAT, Banca d'Italia, BCE, ONU, FMI.
	\section{Le misurazioni economiche}
	\paragraph{Alcune definizioni}
	\begin{itemize}
		\item \textbf{Beni e servizi finali:} beni e servizi venduti al consumatore
		\item \textbf{Beni e servizi intermedi:} beni e servizi scambiati tra le imprese, che diventano fattori di produzione
		\item \textbf{Valore aggiunto:} differenza tra valore delle vendite e valore dei fattori di produzione
	\end{itemize}
	\section{Prodotto interno lordo (PIL)}
	Il valore totale di tutti i beni e servizi finali prodotti da un sistema economico.
	\begin{itemize}
		\item \textit{prodotto:} produzione di beni per il mercato;
		\item \textit{interno:} prodotti sul territorio nazionale;
		\item \textit{lordo:} include tutto, anche il deprezzamento dei mezzi di produzione (ammortamento)
	\end{itemize}
	\subsection{Misure alternative della ricchezza}
	\begin{itemize}
		\item \textbf{Prodotto nazionale lordo (PNL):} reddito totale ottenuto dai fattori di produzione nazionale localizzati anche all'estero
		\item \textbf{Prodotto interno lordo (PIL):} ottenuto dai fattori localizzati in Italia anche se esteri
	\end{itemize}
	\subsection{Calcolo del PIL}
	Esistono tre metodi:
	\begin{enumerate}
		\item \textit{PIL come valore della produzione di beni e servizi \underline{finali}:} si include nel calcolo solo il valore aggiunto di ciascun produttore
		\item \textit{PIL come spesa per l'acquisto di beni e servizi finali prodotti dalle imprese nazionali:} si conteggia solo il valore delle vendite
		\item \textit{PIL come il reddito dei fattori corrisposto dalle imprese nel sistema economico:} reddito lordo da lavoro, da capitale e imposte indirette (IVA)
	\end{enumerate}
	\subsection{PIL reale e nominale}
	Il PIL misura il valore di beni e servizi prodotti in un anno. Il PIL nominale lo misura a prezzi correnti, mentre quello reale utilizza i prezzi di un \textit{anno base}.
	\medskip \\
	Il benessere viene correttamente misurato dal nuovo prodotto in termini reali.
	\subsubsection{Il PIL reale tiene conto dell'inflazione}
	Le variazioni del PIL nominale sono dovute a variazioni di quantità o prezzi. Isolando queste variazioni otteniamo il PIL reale.
	\subsection{PIL pro capite}
	Valore del PIL diviso per la popolazione del paese, è però un indicatore imperfetto perché non considera i beni e servizi non di mercato (es. auto produzione), la qualità dell'ambiente e la distribuzione del reddito.
	\section{Indice dei prezzi e tasso di inflazione}
	\subsection{Alcune definizioni}
	\begin{itemize}
		\item \textbf{Livello generale dei prezzi:} misura del livello complessivo dei prezzi
		\item \textbf{Paniere di mercato:} insieme ipotetico di beni e servizi acquistati dal consumatore medio
		\item \textbf{Indice dei prezzi:} costo dell'acquisto di un dato paniere in un dato anno
		\item \textbf{Indice dei prezzi al consumo:} costo di un paniere rappresentativo dei consumi della famiglia media residente in aree urbane
		\item \textbf{Tasso di inflazione:} variazione percentuale annua di un indice dei prezzi
	\end{itemize}
	\subsection{L'inflazione}
	Processo di aumento continuo e generalizzato del livello dei prezzi di beni destinati al consumo delle famiglie.\medskip \\Un suo aumento corrisponde ad un aumento di velocità della crescita dei prezzi, mentre una sua riduzione ad un aumento ma di velocità minore.\medskip \\
	L'ISTAT produce tre indici dei prezzi al consumo:
	\begin{itemize}
		\item \textbf{per l'intera collettività nazionale (NIC):} misura l'inflazione a livello dell'intero sistema economico
		\item \textbf{per le famiglie di operai e impiegati (FOI):} riferito ai consumi delle famiglie che fanno capo a un lavoratore dipendente. Usato per adeguare periodicamente i valori monetari
		\item \textbf{Indice armonizzato europeo (IPCA):} misura l'inflazione comparabile a livello europeo.
	\end{itemize}
	\section{La disoccupazione}
	\subsection{Concetti base}
	\begin{itemize}
		\item \textit{Popolazione in età lavorativa (WAP)}
		\item \textit{Forza lavoro (FL):} somma di occupati e disoccupati
		\item \textit{Tasso di partecipazione alla forza lavoro (TdP):} percentuale della popolazione in età lavorativa che fa parte della forza lavoro
		\item \textit{Tasso di disoccupazione (TdD):} percentuale di forza lavoro disoccupata
	\end{itemize}
	\begin{figure}[h]
		\centering
		\includegraphics[width=0.7\linewidth]{images/flussi-mercato-lavoro}
		\caption{Flussi nel mercato di lavoro}
		\label{fig:flussi-mercato-lavoro}
	\end{figure}
	\subsection{Problemi di misurazione della disoccupazione}
	I lavoratori scoraggiati, ovvero che hanno rinunciato a cercare lavoro viste le condizioni del mercato, non fanno parte di FL quindi non vengono calcolati in DIS.
	\medskip \\I sottoccupati: lavoratori part-time perché non trovano lavori a tempo pieno.
	\medskip \\Coloro che dichiarano di essere disoccupati per ricevere il sussidio anche se in realtà hanno un lavoro o non lo cercano veramente.
	\subsection{Tipi di disoccupazione}
	In base alla prospettiva temporale si distingue:
	\begin{itemize}
		\item DIS di lungo periodo
		\item DIS di breve periodo
	\end{itemize}
	In base alla sua natura:
	\begin{itemize}
		\item DIS strutturale (lungo periodo)
		\item DIS frizionale (breve e lungo periodo)
		\item DIS ciclica (breve periodo)
	\end{itemize}
	\paragraph{Disoccupazione frizionale} Deriva dalla durata e dalle imperfezioni nel processo di abbinamento tra lavoro e lavoratore. Considerata DIS volontaria.
	\medskip \\Considerata di breve periodo, ma può diventare di lungo periodo a casa delle inefficienze e ostacoli nel mondo del lavoro.
	\paragraph{Disoccupazione strutturale} Prodotta quando il numero di individui alla ricerca di lavoro supera il numero di posti disponibili, eccesso di offerta di lavoro.
	\medskip \\La DIS frizionale da ricerca deriva dal fatto che occorre tempo per trovare il lavoro adatto a un lavoratore. Possono esserci poi problemi di collocamento, nella formazione o scarsa mobilità sul territorio quindi un certo ammontare di DIS è inevitabile e naturale.
	\section{Moneta e sistema bancario}
	La moneta è l'insieme dei valori di un'economia che gli agenti usano per acquistare beni. La sua definizione non include tutte le forme di ricchezza.
	\subsection{Le funzioni della moneta}
	\paragraph{Mezzo di scambio} Qualsiasi cosa che sia accettata normalmente come corrispettivo di beni. è la funzione che più propriamente caratterizza un valore come moneta.
	\medskip \\La liquidità indica la facilità con cui un valore può essere convertito in un mezzo di scambio.
	\paragraph{Unità di conto} Termine di riferimento per determinare prezzi e registrare debiti.
	\paragraph{Riserva di valore} Valori che possono essere usati per trasferire potere d'acquisto dal presente al futuro. Generalmente la liquidità è inversamente correlata alla riserva di valore.
	\subsection{La domanda di moneta}
	\paragraph{Domanda di moneta  a scopo transattivo} Si domanda moneta per utilizzarla come \underline{mezzo di scambio}. Proporzionale al valore delle transazioni da effettuare e dal grado di sincronizzazione (i ricavi e le spese non sono sincronizzati, si desidera detenere moneta).
	\paragraph{Domanda di moneta a scopo speculativo o precauzionale} Si domanda moneta per utilizzarla come riserva di valore. La domanda cresce al diminuire del tasso di interesse.
	\section{La Banca Centrale}
	Istituzione deputata alla supervisione del sistema bancario e al controllo della quantità di moneta nel sistema. Frutto della scelta del policy-maker di conferire il monopolio di emissione di moneta a un singolo istituto.
	\medskip \\La moneta quindi non viene prodotta in libera concorrenza ma si preferisce tutelare la stabilità monetaria; si garantisce che non dipenda da un "sovrano"; solitamente si assegna alla BC la gestione dell'emissione di moneta per limitare l'inflazione.
	\subsection{Le funzioni della Banca Centrale}
	\paragraph{Politica monetaria} Azioni volte a determinare l'offerta di moneta. \medskip \\La BCE delega le Banche centrali dei Paesi nella produzione di euro.
	\paragraph{Vigilanza} La BC regole e controlla l'attività delle banche al fine di promuovere il regolare e sicuro funzionamento del sistema.
	\paragraph{Gestione del sistema di pagamenti} Agisce da stanza di compensazione nei rapporti tra banche
	\paragraph{Banca dello Stato} Gestisce il c/c del Ministero del Tesoro, quindi tutti gli incassi e pagamenti dello stato.
	\paragraph{Banca delle banche} Concede regolarmente prestiti alle banche che la usano come fonte di liquidità alternativa ai depositi dei clienti.
	\subsection{Il sistema europeo di banche centrali e l'Eurosistema}
	Il SEBC è costituito dalla BCE e dalle Banche Centrali Nazionali degli Stati membri. L'Eurosistema invece comprende solo le BCN degli Stati adottanti l'euro.
	\medskip \\Le decisioni di politica monetaria della BCE sono assunte dal consiglio direttivo. L'obiettivo principale dell'Eurosistema è il mantenimento della stabilità dei prezzi. Mario Draghi è l'attuale presidente della BCE.
	\section{Il settore bancario}
	L'attività principale di una banca è quella di raccogliere liquidità nel sistema economico e prestarla a coloro che ne hanno bisogno. Ogni prestito comporta un rischio di credito.
	\medskip \\Le riserve delle banche sono costituite dalla moneta disponibile in banca  necessaria a far fronte alle richieste di prelievo.
	\medskip \\Lo spread dei tassi è la differenza che esiste tra il tasso di interesse che una banca applica sulla liquidità prestata e il minor tasso di interesse che la banca paga sui depositi dei risparmiatori. Permette alla banca di ripagare i servizi bancari.
	\subsection{Fiducia e corsa agli sportelli}
	La fiducia degli agenti della solidità del sistema bancario è essenziale per il suo corretto funzionamento. La banca non tiene a riserva che una piccola frazione dei depositi: il resto viene impiegato in prestiti e investimenti.
	\medskip \\Impedire che una crisi di fiducia possa propagarsi a tutte le banche è uno dei compiti della Banca Centrale, che deve difendere a tutti i costi la fiducia nel sistema ad esempio agendo come prestatore di ultima istanza (le banche però potrebbero azzardare prestiti, tanto interviene mamma BC a salvarla dal fallimento).
	\section{Il sistema finanziario}
	Costituito da diverse istituzioni con il compito di trasferire le risorse scarse dai risparmiatori ai prenditori. La sua attività è quindi quella di intermediazione e coordinamento.
	\medskip \\
	Tipi di istituzioni finanziarie:
	\begin{itemize}
		\item \textbf{Mercati finanziari:} istituzioni dove risparmiatori entrano direttamente in contatto con i prenditori. Es. mercato azionario
		\item \textbf{Intermedi finanziari}: risparmiatori e prenditori sono messi in contatto in via indiretta. Es. banche
	\end{itemize}
	\subsection{Mercato obbligazionario}
	\paragraph{Obbligazione} Titolo di credito che specifica le obbligazioni del debitore nei confronti di chi detiene il titolo. Il rendimento è dato dall'interesse e dal guadagno in conto capitale (aumento di valore nel tempo).
	\medskip \\I titoli obbligazionari presentano 3 caratteristiche:
	\begin{enumerate}
		\item \textbf{Durata}
		\item \textbf{Rischio di credito}, ovvero la probabilità che un debitori non onori gli impegni presi. Si può valutare il rischio di credito ricorrendo a diverse agenzie di rating.
		\item \textbf{Relazione tra il prezzo di un'obbligazione e il suo rendimento}
	\end{enumerate}
	\subsection{Mercato azionario e azione} Rappresenta un titolo di proprietà di un'impresa quindi costituisce un diritto sui profitti che questa realizza. Implicano rischi maggiori ma anche un rendimento potenzialmente illimitato.
	\subsection{Intermediari finanziari}
	Le banche svolgono una funzione di intermediazione, prendendo i depositi e trasformandoli in prestiti. Hanno anche una fondamentale funzione monetaria creando mezzi di scambio. Il loro utile deriva dal margine di intermediazione, ovvero la differenza tra tassi attivi imposti ai debitori e tassi passivi concessi ai depositanti.
	\medskip \\Il fondo comune di investimento è un'istituzione che vende quote di se stessa al pubblico e utilizza il ricavato per comprare un portafoglio di titoli di vario tipo. Consente anche a chi ha pochi soldi di diversificare gli investimenti. La gestione dei risparmi è affidata a professionisti.
	\section{Macroeconomia vs microeconomia}
	Milioni di azioni individuali, accomunandosi, producono un risultato maggiore della loro semplice somma.
	\paragraph{Paradosso della parsimonia} Quando gli agenti temono di dover affrontare un periodo di crisi riducono le spese, deprimendo l'economia consumando meno quindi costringendo le imprese a licenziare i lavoratori. La prudenza che le induce a ridurre le spese amplifica la tendenza negativa del sistema.
	\subsection{Il ciclo economico}
	Alternanza di recessioni (periodi di diminuzione di produzione e occupazione) ed espansioni (periodi di rialzo dell'economia).\\
	
	\chapter{Il bilancio}
	\paragraph{Esercizio} Unità economica e temporale di cui si ritiene significativo determinare i risultati della trasformazione economica.
	\medskip \\I risultati non hanno significato compiuto se non si considera anche la struttura del capitale a inizio esercizio.
	\medskip \\Per operare efficacemente le organizzazioni hanno bisogno di conoscere quante risorse stanno impiegando nelle diverse attività e se il loro utilizzo è economicamente conveniente. Le stesse informazioni sono necessarie agli attori esterni per esprimere un giudizio sulle organizzazioni e legittimarle.
	\paragraph{Contabilità} Processo di rilevazione, misurazione, analisi, interpretazione e comunicazione di informazioni che consentono agli utilizzatori di formulare giudizi e valutazioni sull'impresa.
	\begin{figure}[h]
		\centering
		\includegraphics[width=0.7\linewidth]{images/classificazione-informazioni}
		\caption{Classificazione delle informazioni aziendali}
		\label{fig:classificazione-informazioni}
	\end{figure}
	\section{Le informazioni} I sistemi contabili trattano informazioni quantitative, espresse in termini monetari.
	\paragraph{Informazioni contabili operative} Principale fonte di dati elementari per la formazione del bilancio, per fornire informazioni al management e per poter pagare le imposte.
	\paragraph{Informazioni di bilancio} Usate sia dal management che da terzi. Le regole di redazione del bilancio sono date dal codice civile e dalla prassi contabile.
	\paragraph{Informazioni per il management} Informazioni utilizzate nell'ambito delle tipiche funzioni di management (programmazione).
	\paragraph{Informazioni fiscali} Informazioni necessarie per determinare il reddito civilistico, diverso da quello sulle imposte.
	\section{Principi contabili}
	\subsection{Criteri per la formulazione dei principi contabili}
	\begin{enumerate}
		\item \textbf{Rilevanza:} se produce informazioni importanti e utili riguardo a un impresa
		\item \textbf{Oggettività:} se produce informazioni non influenzate dal giudizio di chi le fornisce
		\item \textbf{Fattibilità:} se può essere implementato senza eccessivi costi o complessità.
	\end{enumerate}
	\subsection{Fonti dei principi contabili}
	In Italia la prassi contabile è definita dall'Organismo italiano di Contabilità. A livello internazionale criteri e principi sono emanati dallo IASB e sono denominati \textbf{IFRS} (International Financial Reporting Standard).
	\section{I rendiconti economico-finanziari}
	La finalità ultima del processo contabile è la produzione di documenti o rendiconti economico-finanziari che riassumono i dati della contabilità sintetizzando i risultati di gestione
	\begin{enumerate}
		\item lo stato patrimoniale
		\item il conto economico
		\item il rendiconto finanziario (non obbligatorio se non quotata in borsa)
		\item la nota integrativa
	\end{enumerate}
	\section{Le funzioni del bilancio}
	Il bilancio (documento amministrativo) svolge tre funzioni:
	\begin{itemize}
		\item intrinseca, di determinazione, del risultato economico prodotto e del correlato capitale di funzionamento, nonché del flusso del cash flow;
		\item estrinseca, di informazione ai terzi dei risultati determinati;
		\item strumentale, di apprezzamento della gestione da parte dei terzi.
	\end{itemize}
	\subsection{Funzione intrinseca}
	Il Bilancio dovrebbe essere formato da tre conti di sintesi:
	\begin{itemize}
		\item Conto Economico
		\item Stato Patrimoniale
		\item Rendiconto Finanziario
	\end{itemize}
	Viene redatto dagli amministratori per rendere conto del loro operato. Il bilancio-rendiconto è tale solo se deriva da una contabilità ben impostata.
	\subsection{Funzione estrinseca}
	Il Bilancio è un documento di informazione sui risultati economici prodotti dalla gestione, destinate ai terzi (soggetti esterni alla black box aziendale).
	\subsection{Bilancio e apprezzamento}
	\paragraph{Apprezzamento} Indagine volta a esprimere un giudizio, favorevole e sfavorevole, sull'impresa in funzionamento, sulle performance gestionali, sulla fitness aziendale.
	\medskip \\Per questo, tutti i lettori del bilancio fanno una prima analisi che consiste nell'interpretazione analitica delle voci, aiutandosi con le informazioni contenute nei documenti, con le loro conoscenze di economia e ragioneria e con informazioni esterne (es. borsa).
	\paragraph{Documento vincolato} Il bilancio non può essere redatto in modo arbitrario, ma deve seguire regole che consentano la determinazione affidabile e l'interpretazione significativa dei dati che contiene. Deve rispettare quindi vincoli tecnico-contabili e vincoli giuridici.
	\section{La rappresentazione contabile dei valori}
	I valori consentono di determinare il risultato economico di un esercizio temporale e lo Stato dei capitali al termine dell'anno stesso.\medskip \\\underline{Risultato economico} e \underline{Stato dei capitali} sono calcolati in due conti di sintesi: lo \textbf{stato patrimoniale}, che espone la sintesi dei valori del capitale investito e quello reperito; il \textbf{conto economico}, che espone la sintesi dei valori della produzione ottenuta e del costo della produzione.
	\subsection{Il sistema dei valori}
	I due conti devono essere sempre formati, letti e analizzati congiuntamente poiché rappresentano il sistema dei valori della trasformazione economica e finanziaria dell'impresa.
	\medskip \\Formano pertanto un micro sistema contabile di valori correlati (se ne varia uno deve per forza variarne almeno un altro).
	\section{I dieci principi del processo contabile}
	\subsection{Il principio del duplice aspetto}
	\[
	\overbrace{\text{Attività}}^\text{Impieghi, o anche risorse economiche} = \overbrace{\text{Passività + capitale netto}}^\text{Fonti finanziarie, o anche\\diritti vantati sulle risorse economiche}
	\]
	\subsection{Principi di omogeneità e identità giuridica}
	La data di riferimento di uno stato patrimoniale può essere qualsiasi.
	\subsection{Principio di omogeneità} La contabilità riporta esclusivamente gli effetti di accadimenti passati esprimibili in termini monetari.
	\subsection{Principio di identità giuridica} L'impresa ha una propria identità giuridica, quindi i rendiconti devono riferirsi all'azienda e non alla proprietà o alle persone.
	\subsection{Il principio della prospettiva di funzionamento}
	Le rilevazioni contabili devono assumere che l'azienda sia in una fase di interruzione dell'attività o che rimanga in vita per un tempo indeterminato (ipotesi più frequente).
	\subsection{Il principio del costo}
	Le risorse economiche di un'azienda sono denominate attività e sono costituite da attività non monetarie (terreni, fabbricati) o monetarie (denaro contante o titoli).
	\medskip \\Un'attività deve essere all'inizio contabilizzata al suo prezzo di acquisto (il costo), che viene cancellato nel tempo in base al consumo stimato del bene.
	\subsection{Principio di prudenza}
	Aspetti principali sono riconoscere i ricavi solo quando sono ragionevolmente certi e riconoscere i costi non appena sono ragionevolmente possibili.\medskip \\è necessario quindi stimare la perdita per crediti inesigibili, rettificando di conseguenza i saldi del capitale netto e dei crediti commerciali. Per questo si utilizza il conto di costo "svalutazione crediti", introducendo nello SP il "fondo svalutazione crediti" che riduce il valore dei crediti commerciali.\medskip \\Quando poi si individuano specifici crediti inesigibili si tolgono dall'ammontare dei crediti e dal fondo svalutazione, senza alcun effetto sui ricavi.
	\subsection{Principio di significatività e rilevanza}
	Bisogna trascurare le transazioni irrilevanti (consumo giornaliero di penne) e individuare le transazioni rilevanti (che se non contabilizzate porterebbero a una valutazione finanziaria dell'azienda differente).\medskip \\Rilevante è una distinzione soggettiva, non definibile attraverso regole precise.
	\subsection{Principio di realizzazione dei ricavi}
	Il ricavo è riconosciuto al momento della consegna. Se viene riconosciuto prima dell'incasso, viene addebitata l'attività crediti commerciali, se l'incasso si verifica prima viene accreditata la passività ricavi anticipati.
	\subsection{Principio di competenza}
	La vendita di merci presenta un aspetto di ricavo (aumento delle riserve di utili) e uno di costo (riduzione delle riserve degli utili non avendo più la merce in magazzino). Entrambi gli aspetti devono essere riconosciuti in uno stesso periodo.
	\section{Lo stato patrimoniale}
	\subsection{Le aree dello stato patrimoniale}
	\begin{itemize}
		\item \textbf{Attività:} risorse economiche possedute dall'azienda che hanno un valore economico e dalle quali ci si attende un beneficio futuro.
		\hr
		\item \textbf{Passività e capitale netto:} fonti di finanziamento per l'impresa che hanno consentito l'acquisto delle attività.
		\medskip \\Sono debiti nei confronti di terzi. I creditori hanno diritti prioritari sulle attività pari ai loro crediti e possono chiedere il fallimento dell'azienda se non onora i debiti.
		\item \textbf{Capitale netto:} fonti di denaro apportate direttamente dalla proprietà e dalle riserve di utili generati attraverso la gestione e non distribuiti sotto forma di dividendo.
	\end{itemize}
	\subsection{Attività correnti e immobilizzate}
	\begin{figure}[h]
		\centering
		\caption{Un modo di classificare le attività}
		\label{fig:classificazione-attivita}
		\includegraphics[width=0.7\linewidth]{images/classificazione-attivita}
	\end{figure}
	\begin{itemize}
		\item Attività correnti:
		\begin{itemize}
			\item liquidità vere e proprie
			\item attività che si presume si trasformeranno in liquidità entro un anno \\
		\end{itemize}
		\item Attività immobilizzate:
		\begin{itemize}
			\item immobilizzazioni materiali
			\item immobilizzazioni immateriali
			\item immobilizzazioni finanziarie
		\end{itemize}
	\end{itemize}
	\subsubsection{Le attività correnti}
	Comprendono la cassa e quelle attività che ci si aspetta si trasformeranno in cassa entro un anno dalla data di bilancio, come crediti verso clienti, rimanenze di merci o anticipi a fornitori.
	\newpage
	\subsubsection{Le attività immobilizzate}
	\begin{figure}[h]
		\centering
		\includegraphics[width=0.7\linewidth]{images/attivita-immobilizzate}
		\label{fig:attivita-immobilizzate}
	\end{figure}
	\paragraph{L'avviamento} Differenza tra prezzo d'acquisto e capitale netto, dato da attività meno le passività.
	\subsection{Passività correnti e immobilizzate}
	\subsubsection{Passività correnti}
	\begin{itemize}
		\item Finanziarie
		\begin{itemize}
			\item Debiti a breve Vs banche
			\item Quote a breve termine di debiti a M/L termine\\
		\end{itemize}
		\item Di funzionamento o operativo
		\begin{itemize}
			\item Debiti Vs fornitori
			\item Cambiali
			\item Debiti verso il personale
			\item Costi sospesi
		\end{itemize}
	\end{itemize}
	\subsubsection{Passività a lungo termine}
	\begin{itemize}
		\item Prestiti obbligazionari
		\item Mutui passivi
		\item TFR
	\end{itemize}
	\begin{figure}[h]
		\centering
		\includegraphics[width=0.7\linewidth]{images/capitale-netto}
		\caption{Capitale netto, equazione fondamentale del bilancio}
		\label{fig:capitale-netto}
	\end{figure}
	\section{L'analisi delle transazioni}
	\paragraph{Transazione} Qualsiasi evento contabilizzato e qualsiasi transazione che comporta almeno due cambiamenti nello SP. Riguardano sia operazioni con l'esterno che interne.
	\medskip \\
	\begin{itemize}
		\item Qualsiasi transazione influisce su almeno due voci dello stato patrimoniale: partita doppia.
		\item Alcuni eventi non rappresentano transazioni e non sono rilevati dalla contabilità (es. cambiamento valore di mercato dei terreni).
		\item Alcuni eventi influiscono sul capitale netto, come operazioni di vendita.
		\item Altri eventi determinano cambiamenti di attività o passività ma non del capitale, come l'acquisto di merce.
		\item Una vendita presenta un aspetto di ricavo (al momento della vendita) e uno di costo (l'acquisto della merce venduta).
		\item Il reddito di un periodo è dato dalla differenza tra ricavi e costi.
	\end{itemize}
	\subsection{Partita doppia}
	Elementi fondamentali sono il conto, le sezioni dare e avere, le regole di registrazione di incrementi e decrementi.
	\begin{figure}[h]
		\centering
		\includegraphics[width=0.7\linewidth]{images/partita-doppia}
		\caption{Regole della partita doppia}
		\label{fig:partita-doppia}
	\end{figure}
	\begin{figure}[H]
		\centering
		\caption{Bilancio d'esercizio - Schema dell'attivo}
		\includegraphics[width=0.7\linewidth]{images/schema-attivo}
		\label{fig:schema-attivo}
	\end{figure}
	\begin{figure}[H]
		\centering
		\caption{Bilancio d'esercizio - Schema del passivo}
		\includegraphics[width=0.7\linewidth]{images/schema-passivo}
		\label{fig:schema-passivo}
	\end{figure}
	\subsection{I conti e le regole di registrazione degli incrementi}
	Per ogni singola transazione l'importo registrato in dare è sempre pari a quello registrato in avere. \medskip \\Gli incrementi delle attività e dei costi sono registrati in dare del conto corrispondente. Gli incrementi delle passività, del capitale netto e dei ricavi sono registrati in avere. \medskip \\I conti dei ricavi e dei costi sono conti temporanei, vengono azzerati utilizzando il conto "Riserve di utili".\medskip \\I conti delle attività, delle passività e del capitale netto sono conti permanenti i cui saldi sono trasferiti da un periodo contabile al successivo.
	\subsection{Il giornale e il riporto a Mastro}
	Nella pratica, le transazioni sono prima registrate nel giornale e poi riportate a mastro. Nel giornale viene registrato prima l'addebito e poi l'accredito.
	\section{Il conto economico}
	Sintetizza i risultati della gestione, è quindi un rendiconto di flusso. Illustra come il reddito si è formato ed è la differenza tra ricavi e costi. è reso obbligatorio per le società di capitali.
	\medskip \\Alcune caratteristiche:
	\begin{itemize}
		\item Le voci precedute da numeri arabi possono essere ulteriormente suddivise o raggruppate mantenendo la chiarezza; il dettaglio viene indicato nella nota integrativa.
		\item I valori sono al netto di quelli rettificati.
		\item Il codice civile impone che vengano aggiunte voce se quelle indicate non sono sufficienti.
		\item Il reddito viene classificato in 4 aree: area caratteristica, classi A e B; area finanziaria, classi C e D; area straordinaria, classe E; area tributi, voce 22.
		\item I componenti di A e B sono relativi alla \textit{gestione caratteristica o ordinaria}, la loro differenza è significativa e può essere considerata come \textit{risultato operativo caratteristico}.
	\end{itemize}
	\begin{figure}[H]
		\centering
		\caption{Schema civilistico}
		\label{fig:conto-economico}
		\includegraphics[width=1\linewidth]{images/conto-economico}
	\end{figure}
	\newpage
	\subsection{La periodicità della misurazione e le relazioni tra reddito e capitale}
	La contabilità misura il risultato derivante dallo svolgimento di un'attività economica in uno specifico intervallo. Possono essere disposti rendiconti infrannuali relativi a periodi più brevi.
	\medskip \\
	La \underline{contabilità per competenza} misura i ricavi ed i costi riferiti ad un certo periodo amministrativo, cioè il reddito. è più complessa di quella per cassa ma più utile, perché rileva più dati.
	\medskip \\
	Il reddito di un periodo non corrisponde all'aumento di cassa ma una variazione delle riserve di utili. Un aumento del capitale netto, infatti, è un aumento dei diritti vantati dalla proprietà sulle attività aziendali e non è un aumento di cassa.
	\section{I costi}
	\subsection{I costi di competenza}
	I costi di competenza di un periodo, come i ricavi di competenza, non coincidono con i relativi esborsi del periodo. Occorre prima determinare i ricavi del periodo e individuare i costi correlati.
	Esistono tre categorie di costi di competenza:
	\begin{itemize}
		\item Costi associati ai beni consegnati nel periodo, o \textbf{costo del venduto};
		\item Costi associati ad operazioni di gestione svolti nel periodo, o \textbf{costi di periodo}, relativi ad attività amministrative, di marketing...
		\item \textbf{Perdite} di competenza del periodo, dovuti a decrementi di valore delle attività a causa di incendio, furto o altri
	\end{itemize}
	\paragraph{Costo in senso generico} Misura monetaria dell'ammontare di una risorsa utilizzata per un qualche scopo
	\paragraph{Spesa} Riduzione di cassa associata all'acquisto di un'attività. Alla spesa può corrispondere un'attività o un costo di competenza se la merce è stata venduta.
	\paragraph{Esborso} Uscita di cassa, spese per cassa ma anche un qualunque pagamento per cassa come fornitori, rate di mutui o dividendi.
	\medskip \\Durante la vita di un'impresa quasi tutte le spese divengono costi di competenza, però in periodi più brevi non c'è corrispondenza. Per distinguere tra spese  e costi di competenza occorre considerare 4 tipi di transazioni:
	\begin{itemize}
		\item Spese del periodo che sono anche costi di competenza del periodo
		\item Spese sostenute in periodi precedenti che diventano costi di competenza nel periodo (es. merci, costi anticipati, immobilizzazioni tecniche)
		\item Spese del periodo che diventeranno costi di competenza in periodi futuri
		\item Costi di competenza non rilevati da pagare in periodi futuri
	\end{itemize}
	\subsection{Attività che diventano costi di competenza}
	\paragraph{Merci} Diventano costi di competenza quando sono vendute.
	\paragraph{Costi anticipati} Servizi o altre attività non ancora completamente utilizzate. Diventano di competenza nel periodo in cui l'attività è consumata.
	\paragraph{Immobilizzazioni tecniche} Acquistate per essere impiegate nelle future attività per più periodi amministrativi, sicché diventeranno costi di competenza in più periodi (ammortamento).
	\subsection{Perdite, insussistenze e sopravvivenze passive}
	Un'attività diventa costo di quel periodo anche se non ha fornito alcun beneficio, perché il suo valore si consuma. Questi costi, registrati come di competenza, sono chiamati \textbf{perdite} o \textbf{insussistenze passive}.
	\medskip \\La perdita viene registrata anche se è soltanto ragionevolmente possibile, dunque non certa. Se invece l'attività produrrà benefici futuri ma non si può determinare quando, viene considerato costo di competenza del periodo.
	\begin{figure}[h]
		\centering
		\includegraphics[width=0.7\linewidth]{images/sintesi-costi-perdite}
		\caption{Costo del venduto, costi di periodo e perdite}
		\label{fig:sintesi-costi-perdite}
	\end{figure}
	\newpage
	\section{Nota integrativa}
	La funzione della nota integrativa è quella di fornire informazioni integrative, esplicative e complementari ai dati presentati in \textit{SP} e \textit{CE}.\medskip \\La rappresentazione veritiera e corretta della situazione patrimoniale/economica dipende anche da una adeguata informativa della nota integrativa.\medskip \\Allegati tipici sono il rendiconto finanziario, il prospetto movimentazione immobilizzazioni e il prospetto movimentazione patrimonio netto.\medskip \\Obiettivi della nota integrativa sono:
	\begin{itemize}
		\item indicare i criteri di valutazione applicati
		\item illustrare e specificare i valori contenuti in SP e CE e le cause di variazione
		\item fornire informazioni di tipo quantitativo
		\item evidenziare le scelte operate per la stima di alcune voci di bilancio
		\item evidenziare gli effetti delle interferenze fiscali sul bilancio civilistico
	\end{itemize}
	Deve inoltre contenere le seguenti informazioni:
	\begin{itemize}
		\item struttura e natura dell'attività svolta
		\item fatti di rilievo avvenuto dopo la chiusura d'esercizio
		\item informazioni sui rapporti avuti con dirigenti e collegio dei revisori
		\item variazioni intervenute nelle voci dell'attivo e del passivo
		\item numero medio di dipendenti
		\item compensi ad amministratori e sindaci
	\end{itemize}
	\section{Rimanenze e costo del venduto}
	\subsection{Costi di competenza}
	Sono di competenza dell'esercizio i costi relativi a fattori produttivi consumati nell'esercizio generanti ricavo o relativi a fattori non più ritenuti utili per la gestione futura (non recuperabili).
	\medskip \\Per tutti gli altri costi si rende necessario il rinvio, ad esempio fattori produttivi non completamente utilizzati o non ancora correlati con i ricavi.
	\subsection{Le rimanenze dei fattori produttivi}
	Le rimanenze di fattori produttivi possono essere di materie prime, sussidiarie, di consumo, di merci e altri beni destinati alla vendita (rimanenze vere e proprie). Per immobilizzazioni e fattori utilizzati lungo un arco di tempo si parla di ammortamento e risconti.\medskip \\Per le rimanenze vere e proprie prima si individuano le quantità fisiche e successivamente ne si valuta il costo.
	\subsubsection{Quota di ammortamento}
	Il costo delle immobilizzazioni viene imputato a più esercizi mediante il processo di ammortamento. La quota di costo di competenza dell'esercizio è imputata al conto economico,mentre il costo originario è rilevato tra gli elementi attivi del capitale.
	\medskip \\La quota di ammortamento è funzione della vita utile del bene e dipende dalle sue prospettive di utilizzo.
	\subsubsection{Risconti}
	Costi relativi a servizi erogati lungo un arco di tempo che si estende oltre la chiusura d'esercizio, come affitti pagati in anticipo o assicurazioni. La quota di competenza dell'esercizio successivo viene denominata risconto attivo.\medskip \\Questa tecnica può essere applicata anche nei ricavi quando incassati o se incassati in anticipo. La quota di competenza dell'esercizio successivo viene chiamata risconto passivo.
	\subsection{Le rimanenze di produzione}
	La loro determinazione prevede che si determinino i fattori impiegati nella produzione e il loro costo.
	\subsubsection{I tipi puri di impresa}
	\begin{itemize}
		\item \textbf{Impresa commerciale:} vende ai propri clienti la merce come l'ha acquistata dai fornitori. Nello SP il valore delle rimanenze valorizza il costo della merce acquistata ma non ancora venduta.
		\item \textbf{Impresa di produzione:} trasforma le materie prime in prodotti finiti. Il costo del venduto comprende costo dei materiali e costo di trasformazioni. Sono presenti rimanenze di materie prime, di semilavorati e di prodotti finiti.
		\item \textbf{Imprese di servizio:} erogano servizi, cioè beni intangibili. Possono avere rimanenze come carta o per attività svolte per conto dei clienti.
	\end{itemize}
	\subsubsection{Metodi per la misurazione}
	\paragraph{Metodo dell'inventario periodico} Si determina prima l'ammontare delle rimanenze a fine periodo, si calcola il costo del venduto sottraendo dall'ammontare di beni da vendere le rimanenze.
	\paragraph{Metodo dell'inventario perpetuo} Si misura il costo del venduto ogni volta che si consegna e si deduce il livello di rimanenze sottraendo dall'ammontare di beni da vendere il costo del venduto.
	\subsubsection{Conti di rimanenze in un'impresa di produzione}
	\begin{enumerate}
		\item \textbf{Rimanenze di materie prime:} valorizzate al costo di acquisto presenti nelle fatture
		\item \textbf{Rimanenze di semilavorati:} valorizzate come somma delle materie prime utilizzate e costi di trasformazione
		\item \textbf{Rimanenze di prodotti finiti:} beni già portati a completamento ma non consegnati
	\end{enumerate}
	Il costo del prodotto finito è dato da costo dei materiali utilizzati, costo della manodopera, quota adeguata dei costi indiretti di produzione.
	\subsubsection{Costi diretti e indiretti}
	I \textbf{costi diretti} sono quelli riconducibili oggettivamente ai singoli prodotti, il costo relativo è quindi oggettivo, stessa cosa per la manodopera.
	\medskip \\
	Tutti i costi non attribuibili oggettivamente ai singoli prodotti (es. stipendio del direttore, riscaldamento), o che non risulterebbe economicamente conveniente attribuire (es. materiali di consumo), sono detti \textbf{costi indiretti di produzione}.
	\subsubsection{Metodi di valorizzazione delle rimanenze}
	\begin{itemize}
		\item identificazione specifica
		\item Costo ponderato
		\item F.I.F.O. i beni più vecchi sono i primi a essere venduti
		\item L.I.F.O. i beni più recenti sono i primi a essere venduti
	\end{itemize}
	Il principio generale per valorizzare le rimanenze è che debbano essere valorizzate al \underline{minore tra il loro costo e il valore di mercato}. Il margine loro è influenzato dal valore del costo del venduto.
	\section{Ammortamenti}
	\paragraph{Immobilizzazioni tecniche e ammortamento}
	All'atto dell'acquisto un'immobilizzazione tecnica viene registrata al suo costo, comprensivo delle spese di trasporto/installazione o altro. Il costo d'acquisto delle immobilizzazioni viene ripartito sui singoli anni della loro vita utile, ogni anno una parte del costo viene addebitata al costo "ammortamento" e accreditata al "fondo ammortamento".
	\medskip \\\underline{Il valore contabile netto di un'immobilizzazione} è dato dalla differenza tra il suo costo originario e il fondo di ammortamento, non è dunque il suo valore di mercato ma il costo che deve ancora essere ammortizzato.
	\subsection{Funzioni dell'ammortamento}
	\begin{enumerate}
		\item \textbf{Funzione patrimoniale:} evidenziare la progressiva perdita di valore degli immobilizzi nello stato patrimoniale
		\item \textbf{Funzione economica:} ripartire un costo di natura pluriennale su più esercizi
		\item \textbf{Funzione finanziaria:} creare disponibilità finanziaria, gli ammortamenti sono un costo fittizio, che non comportano esborsi monetari; sono costi ma non uscite di cassa.
	\end{enumerate}
	\subsection{Calcolo dell'ammortamento}
	\[\text{Quota di ammortamento} = \dfrac{\text{Costo d'acquisto} - \text{Valore residuo}}{\text{vita utile}}\]
	L'ammortamento è una stima. I principali metodi di calcolo sono a quote costanti, accelerato e in base alle unità prodotte.
	\paragraph{Ammortamento a quote costanti} La quota annua di ammortamento si calcola moltiplicando il costo storico per un tasso di ammortamento (1 / anni di vita utile).
	\paragraph{Ammortamento accelerato} Si utilizza quando si ipotizza che il processo di perdita di utilità di una immobilizzazione sia più rapido nei primi anni di vita utile.
	\subsection{L'ammortamento ai fini fiscali}
	Si ha libertà di scelta del criterio di ammortamento, ma bisogna seguire il principio di continuità dei criteri di valutazione. L'ammortamento ai fini fiscali per la determinazione del reddito imponibile:
	\begin{itemize}
		\item Ammortamento ordinario
		\item Ammortamento anticipato
		\item Ammortamento ridotto
		\item Ammortamento integrale
	\end{itemize}
	\subsection{Alienazione di un'immobilizzazione}
	Al momento della vendita di un'immobilizzazione, la differenza tra prezzo di vendita e valore contabile netto è denominata \underline{plusvalenza} se positiva e \underline{minusvalenza} se negativa e viene rilevata in CE.
	\subsection{Immobilizzazioni immateriali}
	\paragraph{Avviamento} L'ammontare è pari alla differenza tra il prezzo pagato per l'acquisto di un'azienda e il suo capitale netto rivalutato, quota di ammortamento fiscalmente non inferiore a 10 anni.
	\paragraph{Marchi} Quota di ammortamento fiscalmente non inferiore a 10 anni.
	\paragraph{Brevetti} Quota di ammortamento fiscalmente non inferiore a 3 anni.
	\paragraph{Concessioni} Quota di ammortamento deducibile in misura corrispondente alla durata della concessione previsa da contratto o dalla legge.
	\section{Costi e ricavi di futura manifestazione numeraria}
	\paragraph{Costi di futura manifestazione numeraria} Il principio di competenza economica impone di rilevare quei costi di futura manifestazione, anche se solo presunta, ma probabile: costi o quote di futura manifestazione o costi di esistenza probabile di indeterminato ammontare o data.
	\subsection{Ratei}
	\paragraph{Ratei passivi} Costi o quote caratterizzati da uno sfasamento temporale tra l'evento economico e quello finanziario.
	\paragraph{Ratei attivi} Ricavi caratterizzati da uno sfasamento temporale tra l'evento economico e quello finanziario.
	\paragraph{Ratei} L'evento economico anticipa quello finanziario
	\paragraph{Risconti} L'evento finanziario anticipa quello economico
	\begin{figure}[h]
		\centering
		\includegraphics[width=0.7\linewidth]{images/ratei-risconti}
		\caption{Ratei e risconti}
		\label{fig:ratei-risconti}
	\end{figure}
	\paragraph{Accantonamento ai fondi} Componenti negativi di reddito e come tali imputati a CE; hanno come correlata variazione finanziaria passività di esistenza certa o probabile, ma indeterminata  nell'ammontare o nella data di sopravvivenza (es. manutenzioni programmate).
	\section{Rendiconto dei flussi di cassa}
	Riporta le entrate e le uscite di cassa relative a un periodo amministrativo. Un'azienda in espansione o in crisi porta maggiore attenzione al rendiconto piuttosto che al CE.
	\medskip \\Spiega il cambiamento nel periodo della liquidità, cioè quali sono state le fonti di liquidità e come è stata consumata.
	\subsection{Attività tipiche che hanno effetto sulla cassa}
	\paragraph{Attività operative} Danno origine a costi e ricavi
	\paragraph{Investimenti e alienazioni} Acquisto e dismissioni di immobilizzazioni tecniche e di altre attività a lungo termine.
	\paragraph{Attività finanziarie} Includono l'ottenimento di fonti di finanziamento, il rimborso di debiti e il pagamento di dividendi.
	\begin{figure}[H]
		\centering
		\includegraphics[width=0.55\linewidth]{images/image-2}
		\caption{Esempi di attività tipiche}
		\label{fig:image-2}
	\end{figure}
	\section{Certificazione del bilancio}
	La documentazione contabile e i bilanci sono controllate da apposite società di certificazione indipendenti, iscritte in un particolare albo. Il processo è denominato certificazione di bilancio (auditing) svolto dai revisori.
	\medskip \\La società di certificazione redige un documento ove esprime un parere sintetico sul bilancio. Se lo esprime senza riserva:
	\begin{itemize}
		\item affermano di aver preso visione dei dati di responsabilità dell'azienda;
		\item dichiarano che le voci e il loro ammontare rappresentano adeguatamente i risultati economici e finanziari conseguiti;
		\item dichiarano che il bilancio è conforme ai principi contabili e alle norme di legge.
	\end{itemize}
	Qualora la situazione non consenta di esprimere in forma positiva anche uno solo di questi pareri viene espresso un parere con riserva, evidenziano i punti non conformi.
	\chapter{Analisi di bilancio}
	Il bilancio d'esercizio è lo strumento fondamentale per l'apprezzamento da parte dei terzi, in modo da esprimere un giudizio sull'\textbf{impresa in funzionamento}, sulle \textbf{performance gestionali} conseguite e sulla \textbf{fitness aziendale}.
	\medskip \\
	Le analisi di bilancio non esprimono un giudizio, ma sono strumentali per ottenerlo. La \textbf{performance} consente di apprezzare la gestione svolta, mentre la \textbf{fitness} permette di giudicare le potenzialità future della struttura patrimoniale finanziaria per produrre risultati economici.
	\medskip \\
	L'apprezzamento è un momento distinto e successivo a quello della determinazione di risultati e performance. Il processo di apprezzamento non è mai oggettivo e neutrale.
	\section{La procedura di massima}
	\paragraph{1. Verifica dei presupposti} L'analista deve poter operare su un bilancio significativo (normale, che riflette la realtà, redatto secondo i principi corretti e consistente nel tempo) e per un'impresa significativa (industriale o mercantile, in normale funzionamento, unitaria, economicamente autonoma e indipendente).
	\paragraph{2. La riclassificazione} Fase più importante in cui l'analista deve sfoltire e raggruppare in poche classi significative, rettificare i valori non significativi, calcolare nuovi valori ed effettuare le prime analisi di struttura.
	\paragraph{3. Calcolo dei ratios} La riclassificazione è propedeutica al calcolo degli indici, divisi in indici quantitativi, indicatori derivanti dai flussi di fondi, indici compositi e dati non quantitativi.
	\paragraph{4. Tipologia di giudizi} Vengono elaborati diversi tipi di analisi specializzate per esprimere giudizi sulla liquidità, sulla solidità ed elasticità, sulla redditività e sull'economicità.
	
	\section{Riclassificazione e Margini dello stato patrimoniale}
	Lo stato patrimoniale dovrebbe dare evidenza al capitale dell'impresa a fine esercizio, che può essere definito economicamente (complesso dei fattori disponibili utilizzabili in futuro) e finanziariamente (complesso degli investimenti e finanziamenti osservabili e valutabili).
	\medskip \\Si ricorda che la struttura produttiva è formata da fattori di struttura (macchinari, impianti) e scorte (rimanenze). Ogni acquisto di fattori che formano la struttura è un investimento di capitale. Gli investimenti hanno bisogno di capitali (prestito, capitale proprio o autofinanziamento).
	\subsection{La struttura del capitale}
	\[CI = KL + CL + MAG + IMM = FO + FP + PN = CR\]
	In qualunque istante, il capitale investito uguaglia il capitale reperito (CI = CR).
	\medskip \\
	Raggruppando i valori liquidi si determinano le liquidità totali:
	\[ LT = KL + CL - FO\]
	\[CCN = LT + MAG - FB\]
	\[CCN + IMM = FL + PN\]
	\subsection{Riclassificazione dello stato patrimoniale}
	Il fine della riclassificazione dello SP è quello di far emergere la struttura dei valori precedentemente indicati e consentire di ottenere sintesi successive. Si utilizza il criterio finanziario del grado di liquidità decrescente.
	\subsubsection{Classi di liquidità dell'attivo}
	\begin{enumerate}
		\item \textbf{Classe delle disponibilità o attività correnti:} poste con liquidità tale da poter essere convertite in denaro entro 12 mesi
		\begin{enumerate}[label=\alph*)]
			\item \textit{liquidità immediate}
			\item \textit{liquidità differite} (a breve termine)
			\item \textit{realizzabilità}: convertite in denaro solo a seguito di operazioni di realizzo
		\end{enumerate}
		\item \textbf{Classe delle immobilizzazioni:} comprende le poste dotate di ridotto grado di liquidità
		\begin{enumerate}[label=\alph*)]
			\item \textit{immobilizzazioni tecniche materiali nette:} fattori produttivi pluriennali materiali
			\item \textit{immobilizzazioni tecniche immateriali nette:} brevetti, marchi avviamento
			\item \textit{immobilizzazioni finanziarie o patrimoniali nette:} finanziamenti attivi a lungo termine
		\end{enumerate}
	\end{enumerate}
	I valori devono essere già al netto dei valori rettificativi, detraendo eventualmente quelli nel passivo.
	\medskip \\I valori delle passività possono essere riclassificati secondo il criterio del grado decrescente di liquidazione (attitudine  a trasformarsi in denaro, con eventuali oneri).
	\subsubsection{Classi di liquidità del passivo}
	\begin{enumerate}
		\item \textbf{Passività correnti:} con cadenza supposta non superiore all'anno (fornitori, cambiali passive ecc.)
		\item \textbf{Passività consolidate:} debiti con durata pluriennale (TFR, mutui ecc.)
	\end{enumerate}
	La riclassificazione dello SP a liquidità decrescente, oltre a fornire una prima informazione sui capitali che l'impresa ha investito e reperito, consente di cogliere la coerenza complessiva tra la composizione di attivo e passivo e l'equilibrio tra la partecipazione degli azionisti al finanziamento dell'impresa e il concorso degli altri portatori di capitali.
	\newline
	\begin{figure}[h]
		\centering
		\includegraphics[width=0.7\linewidth]{images/sp}
		\caption{Stato patrimoniale classificato finanziariamente}
		\label{fig:sp}
	\end{figure}
	\subsection{Le fonti onerose di finanziamento}
	Le fonti onerose di finanziamento di un'azienda sono il debito finanziario e il capitale netto, che finanziano rispettivamente il capitale circolante operativo netto e le attività immobilizzate nette.
	\paragraph{Il debito finanziario} è composto da debiti di finanziamento a breve (c/c) e lungo termine (mutui, prestiti, leasing). Gli impegni del debito sono rimborsare il valore del debito e pagare gli interessi.
	\paragraph{Il capitale netto} è composto da capitale versato, riserve di sovrapprezzo azioni, azioni ordinarie, capitale sociale, azioni proprie, capitale in circolazione e valore di mercato/di bilancio del capitale netto.
	\begin{figure}[h]
		\centering
		\includegraphics[width=0.7\linewidth]{images/confronto-capitali}
		\caption{Confronto tra capitale netto e debito}
		\label{fig:confronto-capitali}
	\end{figure}
	\subsection{Quattro margini fondamentali}
	\subsubsection{Margine di tesoreria}
	\[MT = LI + LD - PC\]
	LI = liquidità immediate\\LD = liquidità differite\\PC = passività circolanti
	\subsubsection{Margine di disponibilità o Capitale circolante netto}
	\[MD = CCN = AC - PC\]
	AC = attività correnti\\PC = passività correnti
	\subsubsection{Margine di struttura}
	\[MS = PN - AF\]
	PN = patrimonio netto\\AF = immobilizzazioni nette (o attività fisse)
	\subsubsection{Capitale investito netto e Capitale reperito netto}
	\[CIN = CCN + AF = PF + PN = CRN\]
	PF = passività fisse
	\subsection{Tre postulati di buona gestione}
	\paragraph{Postulato 1} Il margine di tesoreria deve essere positivo
	\paragraph{Postulato 2} Il capitale circolante netto deve essere almeno pari alle passività correnti
	\paragraph{Postulato 3} Il margine di struttura deve essere positivo
	\subsection{Tre postulati di buona dinamica}
	\paragraph{Postulato 4} L'indebitamento (FP) rispetto all'attivo (ATT) non deve aumentare nel tempo
	\paragraph{Postulato 5} La dinamica del CCN deve essere congrua con quella del Valore della produzione. L’aumento troppo accentuato del CCN rispetto al VdP sarebbe un indicatore o di un allungamento dei tempi di incasso o di un appesantimento di magazzino. Viceversa se CCN aumentasse in misura troppo modesta, significherebbe un aumento sproporzionato di debiti verso i fornitori o una riduzione troppo spinta delle scorte di prodotti finiti.
	\paragraph{Postulato 6} Il rapporto tra debiti (FP) e fatturato o valore della produzione VdP non deve aumentare nel tempo (meglio se in riduzione)
	
	\section{Riclassificazione del conto economico}
	Si basa sull'individuazione di diversi saldi intermedi che mettono in evidenza l'incidenza delle diverse gestioni.
	\paragraph{Valore aggiunto lordo} Differenza tra il valore della produzione e i costi esterni della produzione di tipo cash
	\paragraph{Margine operativo lordo} Margine di reddito realizzato sul costo delle risorse produttive interne ed esterne all'impresa.
	\paragraph{Margine operativo netto MON/EBIT} Margine di reddito realizzato sul costo di tutte le risorse produttive e del capitale immobilizzato.
	\begin{figure}[H]
		\centering
		\includegraphics[width=0.7\linewidth]{images/ce-costo-venduto}
		\caption{Riclassificazione del Conto Economico a costo del venduto}
		\label{fig:ce-costo-venduto}
	\end{figure}
	\begin{figure}[H]
		\centering
		\includegraphics[width=0.7\linewidth]{images/ce-valore-aggiunto}
		\caption{Riclassificazione del Conto Economico a valore aggiunto}
		\label{fig:ce-valore-aggiunto}
	\end{figure}
	\subsection{Valore aggiunto}
	Rappresenta l'incremento attribuito ai beni e ai servizi acquistati all'esterno per effetto del processo produttivo.\\Differenza tra i ricavi ottenuti vendendo beni e servizi prodotti e i costi sostenuti per acquistare all'esterno i beni e servizi necessari per ottenerli.\\Corrisponde alla somma dei redditi o compensi attribuiti a vario titolo a tutti i soggetti che partecipano alla produzione.
	\section{Indici di bilancio}
	Suddivisi in indici di struttura finanziaria (o solidità), indici di liquidità, indici di rotazione e indici di redditività.
	\subsection{Indici di struttura finanziaria}
	Consentono di valutare la struttura finanziaria dell'impresa con particolare attenzione all'equilibrio fra capitale proprio e di terzi.
	\subsubsection{Leverage D/E}
	Rapporto tra debiti finanziari e patrimonio netto. Valori corretti sono tra 0,5 e 2
	\[D/E = \frac{\text{Debiti finanziari}}{\text{Capitale netto}}\]
	\subsubsection{Indice di indebitamento}
	Analoga informazione, rapporto tra debiti finanziari e capitale investito
	\[DR = \frac{\text{Debiti finanziari}}{\text{Capitale investito}}\]
	\subsubsection{Indice di indipendenza finanziaria}
	\[indice = \frac{MP}{MP+MT}\]
	MP = mezzi propri\\MT = mezzi di terzi\\minore di 0,33 struttura finanziaria critica\\0,33 - 0,55 struttura da monitorare\\0,55 - 0,66 struttura equilibrata\\maggiore di 0,66 possibilità di sviluppo
	\subsection{Indici di liquidità}
	Segnalano la capacità dell'impresa a far fronte ai propri impegni di cassa nel breve periodo.
	\subsubsection{Liquidità corrente}
	\[CR = \frac{\text{Attività correnti}}{\text{Passività correnti}}\]
	Se inferiore a 1 le attività correnti non coprono il fabbisogno di liquidità nel breve termine.
	\subsubsection{Test acido (o quick ratio)}
	\[QR = \frac{\text{Attività correnti - Rimanenze}}{\text{Passività correnti}}\]
	Non considera le rimanenze di magazzino come fonte di liquidità
	\subsubsection{Quoziente di liquidità immediata (o instant ratio)}
	\[IR = \frac{\text{Cassa + Titoli a breve termine}}{\text{Passività correnti}}\]
	Tiene conto esclusivamente delle attività di immediata liquidazione.
	\begin{figure}[h]
		\centering
		\includegraphics[width=0.7\linewidth]{images/indici-liquidita}
		\caption{Indici di liquidità}
		\label{fig:indici-liquidita}
	\end{figure}
	\subsection{Indici di rotazione}
	\subsubsection{Giacenza media del magazzino}
	\[A = 365 * \frac{\text{Valore del magazzino}}{\text{Costo del venduto nell'esercizio}}\]
	Più è basso meglio l'azienda riesce a trasformare le attività in reddito. Se è uguale a 0 l'azienda riesce a vendere non appena ha terminato di produrre.
	\subsubsection{Durata media dei crediti commerciali}
	\[B = 365 * \frac{\text{Crediti commerciali}}{\text{Fatture di vendita emesse nell'esercizio}}\]
	\subsubsection{Durata media dei debiti commerciali}
	\[C = 365 * \frac{\text{Debiti commerciali}}{\text{Fatture di vendita emesse nell'esercizio}}\]
	\medskip \\
	A+B-C fornisce un'idea sul periodo di esposizione finanziaria dell'impresa.
	\subsubsection{Indice di rotazione dei crediti commerciali}
	\[RC = \frac{\text{Fatturato}}{\text{Crediti commerciali}}\]
	\subsubsection{Indice di rotazione delle rimanenze}
	\[RR = \frac{\text{Costo del venduto}}{\text{Rimanenze di P.F.}}\]
	\subsection{Indici di redditività}
	\subsubsection{ROE (Return on Equity)}
	è un indice della remunerazione che l'impresa genera per gli azionisti, esprime la redditività netta del capitale investito dalla proprietà.
	\[ROE = \frac{\text{Utile netto}}{\text{Capitale netto}}\]
	Viene valutato con confronto longitudinale o su base storica, con riferimenti esterni o con valutazioni soggettive.
	\subsubsection{ROA (Return on Assets)}
	Riflette la redditività di tutte le fonti finanziarie aziendali
	\[ROA = \frac{\text{Risultato operativo}}{\text{Totale attivo}}\]
	\subsubsection{ROI (Return on Investment)}
	Riflette la redditività del capitale investito, ovvero delle fonti finanziare aziendali onerose, o risorse complessive investite.
	\[ROI = \frac{\text{Risultato operativo}}{\text{Debito finanziario + Capitle netto}}\]
	\subsubsection{ROS (Return on Sales o margine sulle vendite)}
	Indice dell'efficienza operativa dell'impresa, esprime la frazione del fatturato che si traduce in margine operativo.
	\[ROS = \frac{\text{Risultato operativo}}{\text{Valore della produzione}}\]
	\subsubsection{RCI (Rotazione del capitale investito)}
	Esprime la frazione di capitale investito che si trasforma in fatturato. Rileva la capacità delle risorse dell'azienda di generare output e misura la produttività del capitale.
	\[RCI = \frac{\text{Valore della produzione}}{\text{Capitale investito}}\]
	\subsubsection{Reddito netto (RNP)}
	Misura dell'efficienza dell'impresa, comprende anche le gestioni finanziaria, patrimoniale e straordinaria e riguarda quindi gli azionisti.
	\[RNP = \frac{\text{Utile netto}}{\text{Valore della produzione}}\]
	\subsubsection{Leva finanziaria}
	Relazione che lega contabilmente il ROE e il ROI. Evidenzia che l'impresa, se è in grado di generare dalle risorse un rendimento superiore al costo del capitale di debito (ROI maggiore di r), può generare valore per gli azionisti incrementando l'indebitamento e facendo leva sul differenziale dei rendimenti.
	\medskip \\Quindi se ROI maggiore di r (costo medio annuo dei debiti) il ROE aumenta (effetto leva), al contrario diminuisce.
	
	
	
	
	
\end{document}
